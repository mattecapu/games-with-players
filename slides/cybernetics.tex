\begin{framecard}
	{\color{white}
	\bfseries

	\hugetext{Cybernetics}}
\end{framecard}

\begin{frame}{Cybernetics}
	From $\kappa\upsilon\beta\varepsilon\rho\nu\alpha\omega$ (\emph{to govern, to steer}):
	\vfill

	\begin{quotation}
		Science concerned with the study of systems of any nature which are capable of receiving, storing and processing information so as to use it for control.
		{\color{colornote}-- Kolmogorov}
	\end{quotation}

	Lenses model dynamical systems (see: \cite{myers2020double})\\
	\textbf{\color{coloraccent}Parametrized lenses (+ decorations) model cybernetic systems!}

	\vfill
	The missing bits are \textbf{storage} and \textbf{feedback}.

	\vfill
	{\color{colornote}
	$\underbrace{\text{Parametrized $\cdots$ parametrized}}_{\text{$n$ times}}$ lenses model \textbf{...?}\\[1.5ex]
		\hspace{5ex} \textbf{Hierarchical agency?}
	}
\end{frame}

\begin{frame}{Higher-order cybernetics}
	Agents
	\begin{enumerate}
		\item act in the arena,
		\item then \textbf<2->{observe} the result of their behaviour,
		\item then change their action accordingly,
	\end{enumerate}
	until an equilibrium is reached.

	\onslide<2->{
	\vfill
	\begin{tabular}{p{.2\textwidth}|p{.12\textwidth}p{.12\textwidth}p{.12\textwidth}p{.1\textwidth}}
		& \textbf{1st} & \textbf{2nd} & \textbf{3rd} & $\cdots$\\
		\hline
		non-trivial\newline observation     &     & yes & yes & $\cdots$\\[2ex]
		\hline
		non-trivial analysis                &     &     & yes & $\cdots$\\[2ex]
		\hline
		$\qquad \vdots$                     &     &     &     & $\ddots$
	\end{tabular}
	}
\end{frame}

\begin{frame}{Periodic table of cybernetic types}
	\vspace{-2ex}
	Reasoning negatively:
	\vspace{2ex}

	\vfill
	\begin{tabular}{p{.16\textwidth}|p{.1\textwidth}p{.1\textwidth}p{.1\textwidth}p{.1\textwidth}p{.1\textwidth}p{.01\textwidth}}
		% static - closed dynamical - open dynamical - cybernetic - 2nd-order cybernetic - ...
		& \textbf{-2nd} & \textbf{-1st} & \textbf{0th} & \textbf{1st} & \textbf{2nd} & $\cdots$\\
		\hline
		{\color{coloraccent}non-trivial system}      &     & yes & yes & yes & yes & $\cdots$\\[2ex]
		\hline
		{\color{coloraccent}non-trivial context}     &     &     & yes & yes & yes & $\cdots$\\[2ex]
		\hline
		{\color{coloraccent}non-trivial interaction} &     &     &     & yes & yes & $\cdots$\\[2ex]
		\hline
		non-trivial observation                      &     &     &     &     & yes & $\cdots$\\[2ex]
		\hline
		$\qquad \vdots$                              &     &     &     &     &     & $\ddots$
	\end{tabular}
\end{frame}

\begin{frame}{Games vs. learners}
	\begin{center}
		\includegraphics[clip, page=16, trim=2cm 5cm 5cm 2cm, width=.8\textwidth]{figures/drawings.pdf}
	\end{center}

	\vfill
	A learner produces its own selection function as a fixpoint:

	\begin{equation*}
		p \in \eq(x,u) \sse p = p \cmp GD \cmp w \cmp (x \cmp \L \cmp u)^\top
	\end{equation*}
\end{frame}

\begin{frame}{2nd order cybernetics}
	Given a parameter $p \in P$, a learner can only observe $\L^\top$ on an infinitesimal neighbourhood of $p$\\
	\qquad $\rightsquigarrow$ \textbf{2nd-order cybernetic systems}

	\vfill
	We can consider $\Para(\Lens(\cat{Smooth}))$ a \textbf{2nd-order \emph{cybernetic doctrine}} (terminology borrowed from \cite{myers2021cst})

	\vfill
	This is actually a strength:
	\begin{enumerate}
		\item We can encode the selection in the parameter dynamics
		\item We can analyze locally and iteratively\\(vs. games `global and one-step')
	\end{enumerate}
\end{frame}
